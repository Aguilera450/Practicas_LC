\documentclass{article}

% Symbols
\usepackage{amsfonts, amsthm}
\usepackage{upgreek}
\usepackage{physics}
\usepackage{cancel}
\usepackage{amssymb, latexsym, amsmath}

%Algorithms
\usepackage[ruled,lined,linesnumbered,commentsnumbered]{algorithm2e}

%% Identación
\setlength{\parindent}{0cm}

% Código
\newcommand{\code}[1]{\textcolor{white!25!black}{\texttt{#1}}}
\usepackage{listings}

%AMS
\usepackage{amsthm}
\newtheorem{algo-thm}{Algoritmo}

% Graphics
\usepackage{graphicx}
\usepackage{pgf}

% Margins
\addtolength{\voffset}{-1.5cm}
\addtolength{\hoffset}{-1.5cm}
\addtolength{\textwidth}{3cm}
\addtolength{\textheight}{3cm}

%Header-Footer
\usepackage{fancyhdr}
\renewcommand{\headrulewidth}{1pt}

\newcommand{\set}[1]{
  \left\{ #1 \right\}
}

\footskip = 50pt
\renewcommand{\headrulewidth}{1pt}

\pagestyle{fancyplain}

\begin{document}
\title{UNIVERSIDAD NACIONAL AUT\'ONOMA DE M\'EXICO\\ Facultad de Ciencias}
\author{Integrantes:\\
  Marco Silva Huerta\\
  Adri\'an Aguilera Moreno\\}
\date{}
\maketitle
\begin{center}
  \includegraphics[scale=0.20]{../Imagen/Portada.jpg}\\[0.4cm]
  \Large
  \bf{Lógica Computacional}
  \normalsize
\end{center}
\newpage
\fancyhead[r]{ Lógica Computacional 2022-2}
%%%%%%%%%%%%%%%%%%%%%%%%%%%%%%%%%%%%%%%%%%%%%%%%%%%%%
\section*{\LARGE{Práctica 1}}
%%%%%%%%%%%%%%%%%%%%%%%%%%%%%%%%%%%%%%%%%%%%%%%%%%%%%%%%%%%%%%%%%%%%%
%%%%%%%%%%%%%%%%%%%%%%% ESPECIFICACIONES AQUÍ %%%%%%%%%%%%%%%%%%%%%%%
%%%%%%%%%%%%%%%%%%%%%%%%%%%%%%%%%%%%%%%%%%%%%%%%%%%%%%%%%%%%%%%%%%%%%

\begin{enumerate}
  
  %%%%%%%%%%%%%%%%%%%%%%%%%% Ejercicio 1
\item \textbf{Anagrama:} Resolvemos ordenando ambas cadenas con \textbf{qsort} y comparando para verificar si tenemos las mismas palabras
  
  %%%%%%%%%%%%%%%%%%%%%%%%%% Ejercicio 2
\item \textbf{Segmento:} Para la función \code{segmento} se emplean dos funciones
  auxiliares, estas son:
  \begin{itemize}
  \item \code{subcadenaInd}: recibe un entero y una lista como parámetros, luego
    regresa una sublista de enteros hasta el índice parámetro (simula take).
  \item \code{eliminaHasta}: recibe un entero y una lista, así esta función elimina
    los elementos en la lista hasta la posición que marca el parámetro (simula drop).
  \end{itemize}
  %%%%%%%%%%%%%%%%%%%%%%%%%% Ejercicio 3
\item \textbf{Multiplica la moda:}

La forma de resolver fue en cinco pasos:
\begin{enumerate}
    \item Ordenar la lista
    \item Hacer que la lista sean tuplas
    \item Ordenar la tupla
    \item Obtener la lista ultima de la tupla 
    \item Multiplicar los números de la lista
\end{enumerate}

\textbf{tsort:} Usamos sortBy para sobreescribir la forma de ordenar con (comparing length), es decir la función tsort nos deja sobrescribir la forma en que se va a ordenar haciendo las coparaciones correspondientes a la longitud de cada una de las listas. Comparando la longitud para hacer el ordenamiento. 

\textbf{tuplas:} usamos takeWhile y dropWhile para hacer las busquedas
 de los números que fueran comparando y cumpliendo la igualdad, rompiendo 
 con los que no de modo que se separarán las listas con los valores iguales. 
  
  %%%%%%%%%%%%%%%%%%%%%%%%%% Ejercicio 4
\item \textbf{Fechas espejo:} Esta función utiliza la función reversa\footnote{Esta
función se define en \code{Practica01} y es, de hecho, la función vista en clase.}
  de una lista como función auxiliar.
  
  %%%%%%%%%%%%%%%%%%%%%%%%%% Ejercicio 5
\item \textbf{Elimina por índice:} Recibimos una lista un número (índice) que buscamos eliminar. Caso base, lista vacía e índice 0. Finalmente hacemos recursión sobre la lista restando al número 1 por iniciar con 0. 
  
  
  
  %%%%%%%%%%%%%%%%%%%%%%%%%% Ejercicio 6
\item \textbf{Suma binarios y antecesor del binario:} Para esta función, en el
  inciso $(a)$\footnote{Antecesor del binario.} se emplea una función auxiliar
  para encontrar el sucesor del complemento $a2$ del binario parámetro, así al
  convertir este valor a su complemento $a2$ nuevamente tendremos el sucesor
  del binario parámetro.
  
  %%%%%%%%%%%%%%%%%%%%%%%%%% Ejercicio 7
\item \textbf{Operaciones:} En esta función se siguen las definiciones de show.
\end{enumerate}
\textbf{Nota:} las ejecuciones son tal cuál las indicadas en la descripción de la práctica.

%%%%%%%% Cuadrito perrón
\begin{center}
  \fbox{
    \begin{minipage}[b][1\height]%
      [t]{0.867\textwidth}
      Matriculas:
      \begin{itemize}
      \item[1.] Marco Silva Huerta: 316205326
      \item[2.] Adrian Aguilera Moreno: 421005200.
      \end{itemize}
  \end{minipage}}
\end{center}

\end{document}
