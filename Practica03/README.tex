\documentclass{article}

% Symbols
\usepackage{amsfonts, amsthm}
\usepackage{upgreek}
\usepackage{physics}
\usepackage{cancel}
\usepackage{amssymb, latexsym, amsmath}

%Algorithms
\usepackage[ruled,lined,linesnumbered,commentsnumbered]{algorithm2e}

%% Identación
\setlength{\parindent}{0cm}

% Código
\newcommand{\code}[1]{\textcolor{white!25!black}{\texttt{#1}}}
\usepackage{listings}

%AMS
\usepackage{amsthm}
\newtheorem{algo-thm}{Algoritmo}

% Graphics
\usepackage{graphicx}
\usepackage{pgf}

% Margins
\addtolength{\voffset}{-1.5cm}
\addtolength{\hoffset}{-1.5cm}
\addtolength{\textwidth}{3cm}
\addtolength{\textheight}{3cm}

%Header-Footer
\usepackage{fancyhdr}
\renewcommand{\headrulewidth}{1pt}

\newcommand{\set}[1]{
  \left\{ #1 \right\}
}

\footskip = 50pt
\renewcommand{\headrulewidth}{1pt}

\pagestyle{fancyplain}

\begin{document}
\title{UNIVERSIDAD NACIONAL AUT\'ONOMA DE M\'EXICO\\ Facultad de Ciencias}
\author{Integrantes:\\
  Marco Silva Huerta\\
  Adri\'an Aguilera Moreno\\}
\date{}
\maketitle
\begin{center}
  \includegraphics[scale=0.20]{../Imagen/Portada.jpg}\\[0.4cm]
  \Large
  \bf{Lógica Computacional}
  \normalsize
\end{center}
\newpage
\fancyhead[r]{ Lógica Computacional 2022-2}
%%%%%%%%%%%%%%%%%%%%%%%%%%%%%%%%%%%%%%%%%%%%%%%%%%%%%
\section*{\LARGE{Práctica 3}}
%%%%%%%%%%%%%%%%%%%%%%%%%%%%%%%%%%%%%%%%%%%%%%%%%%%%%%%%%%%%%%%%%%%%%
%%%%%%%%%%%%%%%%%%%%%%% ESPECIFICACIONES AQUÍ %%%%%%%%%%%%%%%%%%%%%%%
%%%%%%%%%%%%%%%%%%%%%%%%%%%%%%%%%%%%%%%%%%%%%%%%%%%%%%%%%%%%%%%%%%%%%
\newcommand{\localtextbulletone}{\textcolor{black}{\raisebox{.45ex}{\rule{.6ex}{.6ex}}}}
\renewcommand{\labelitemi}{\localtextbulletone}
\begin{itemize}
\item \code{fnn}
    \begin{itemize}
        \item Usamos las funciones: elimImpl, elimEquiv, deMorgan para obetner la forma normal negativa, que estamos eliminando las implicaciones y equivalencias para después aplicar ley de De Morgan y negar hasta que solo variables atómicas lo estén. 
    \end{itemize}
    
\item \code{fnc}
    \begin{itemize}
        \item Aplicamos la forma normal conjuntiva más dos funciones auxiliares, una para que regresa un True si es que encuentra conjunciones y la otra que hace ley de distribución sobre la formula que se le ha pasado
    \end{itemize}
    
\item \code{unit}
    \begin{itemize}
        \item Esta regla consiste en sacar todas las variables libres de una formula 
    \end{itemize}
    
\item \code{elim}

\item \code{red}

\item \code{split}
    \begin{itemize}
        \item La regla consiste en crear dos ramas, una positiva y otra negativa sacando únicamente una variable de la formula en este caso tomamos la primera y hacemos los dos caminos negando la variable de la segunda, hacemos una lista de las formulas y como sabemos que la primera variable de la formula será a la que le hacemos ramas, usando una función \textit{cabeza} obtenemos la letra que después pasamos a variable.  
    \end{itemize}

\item \code{conflict}
\item \code{success}
\item \code{appDPLL}
\end{itemize}

\begin{center}
  \fbox{
    \begin{minipage}[b][1\height]%
      [t]{0.867\textwidth}
      Matriculas:
      \begin{itemize}
      \item[1.] Marco Silva Huerta: 316205326.
      \item[2.] Adrian Aguilera Moreno: 421005200.
      \end{itemize}
  \end{minipage}}
\end{center}

\end{document}
