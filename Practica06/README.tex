\documentclass{article}

% Symbols
\usepackage{amsfonts, amsthm}
\usepackage{upgreek}
\usepackage{physics}
\usepackage{cancel}
\usepackage{amssymb, latexsym, amsmath}

%Algorithms
\usepackage[ruled,lined,linesnumbered,commentsnumbered]{algorithm2e}

%% Identación
\setlength{\parindent}{0cm}

% Código
\newcommand{\code}[1]{\textcolor{white!25!black}{\texttt{#1}}}
\usepackage{listings}

%AMS
\usepackage{amsthm}
\newtheorem{algo-thm}{Algoritmo}

% Graphics
\usepackage{graphicx}
\usepackage{pgf}

% Margins
\addtolength{\voffset}{-1.5cm}
\addtolength{\hoffset}{-1.5cm}
\addtolength{\textwidth}{3cm}
\addtolength{\textheight}{3cm}

%Header-Footer
\usepackage{fancyhdr}
\renewcommand{\headrulewidth}{1pt}

\newcommand{\set}[1]{
  \left\{ #1 \right\}
}

\footskip = 50pt
\renewcommand{\headrulewidth}{1pt}

\pagestyle{fancyplain}

\begin{document}
\title{UNIVERSIDAD NACIONAL AUT\'ONOMA DE M\'EXICO\\ Facultad de Ciencias}
\author{Integrantes:\\
  Marco Silva Huerta\\
  Adri\'an Aguilera Moreno\\}
\date{}
\maketitle
\begin{center}
  \includegraphics[scale=0.20]{../Imagen/Portada.jpg}\\[0.4cm]
  \Large
  \bf{Lógica Computacional}
  \normalsize
\end{center}
\newpage
\fancyhead[r]{ Lógica Computacional 2022-2}
%%%%%%%%%%%%%%%%%%%%%%%%%%%%%%%%%%%%%%%%%%%%%%%%%%%%%
\section*{\LARGE{Práctica 6}}
%%%%%%%%%%%%%%%%%%%%%%%%%%%%%%%%%%%%%%%%%%%%%%%%%%%%%%%%%%%%%%%%%%%%%
%%%%%%%%%%%%%%%%%%%%%%% ESPECIFICACIONES AQUÍ %%%%%%%%%%%%%%%%%%%%%%%
%%%%%%%%%%%%%%%%%%%%%%%%%%%%%%%%%%%%%%%%%%%%%%%%%%%%%%%%%%%%%%%%%%%%%
\newcommand{\localtextbulletone}{\textcolor{black}{\raisebox{.45ex}{\rule{.6ex}{.6ex}}}}
\renewcommand{\labelitemi}{\localtextbulletone}
\begin{itemize}
\item \textbf{Convertidor:}\\
  Todos los casos deberán ser listas, no se acepta hacer una conversión de un símbolo en solitario es decir \textit{char\_binario(a,L).} devuelve false.\\
  Comenzamos entonces por el caso base, si nos pasan una lista vacía para convertirla devolvemos la misma lista vacía.\\
  El segundo paso es ir caso a caso, de modo que toma el único char y lo convierte haciendo recursión sobre los hechos. Para la parte final recorremos toda la lista que nos estén pasando haciendo la recursión en el caso unitario y una vez echa la conversión se agrean a la lista convertida. \\
  Esto es lo mismo para cuando queremos ir de char a binario y de binario a char. 
  
\item \textbf{Cubos:}\\
  \code{sobre} nos indica que cubo esta sobre algún otro, \textit{i.e.}, \code{sobre(X, Y)} indica que \code{X}
  esta sobre \code{Y}.

  \code{hastaArriba} nos indica quién ya no tiene un cubo sobre de el mismo.

  \code{bloqueado} nos dice si el cubo que se pasa como parámetro tiene un cubo arriba de el o no.

  \code{hastaAbajo} nos indica quién es el cubo que esta abajo en la pila de cubos.

  Las funciones auxiliares se especifican como comentario en el programa lógico.
\item \textbf{Autómatas:}\\
  \code{estado\_final} es el estado de aceptación de la lista que se le pasa al autómata si es que esa lista
  es aceptada por el lenguaje que genera el autómata.

  \code{estado\_inicial} este estado es el primero por el que pasa la lista recibida y por donde inicia siempre
  (pues solo hay un estado inicial).

  \code{transicion} es nuestra función de transición donde están definidos todos los cambios de estados.

  \code{termina} esta función nos indica si la lista que se pasa como parámetro es aceptada por el autómata o no,
  después de sus respectivas llamadas recursivas.

  \code{aceptar} es la función que recibe una sola cadena y que por omisión llama a \code{termina} con estado
  inicial $q_0$.
\item \textbf{Mezcla:}\\
  El primer método es un ordenamiento por selección, haciendo uso de las funciones \textit{select} y \textit{member} en el método auxiliar \textit{menor} que se encarga de hacer las todas las comparaciones.\\
  
  Para el método mezcla hay 2 casos base:
  \begin{itemize}
  \item Donde nos pasan dos listas vacías, devolvemos la lista vacía
  \item Donde mezclamos una lista vacía y una lista con elementos (ordenados) darán únicamente la lista ordenada. 
  \end{itemize}
  
  Teniendo esto pasamos a realizar la mezcla tomando en cuenta 3 situaciones:
  \begin{equation*}
    X>Y, \hspace{5mm} X=Y, \hspace{5mm} X<Y
  \end{equation*}
  
  Estas son la situaciones que se nos presentan al comparar las dos listas de modo que el operador de corte (\textbf{!}) se aplica después de hacer dicha comparación. Es decir que si se cumple que $X>Y$ el operador corta las ramas siguientes, caso contrario sigue. \\
  Para poder mezclar, únicamente podemos hacerlo con las dos listas ordenadas así que se creó un método auxiliar \textit{ordenada} que devuelve true o false si es que la lista que le pasan esta ordenada. Aplicándolo al método \textit{mezclar} verifica si estas listas están ordenadas. 
\end{itemize}
\begin{center}
  \fbox{
    \begin{minipage}[b][1\height]%
      [t]{0.867\textwidth}
      Matriculas:
      \begin{itemize}
      \item[1.] Marco Silva Huerta: 316205326.
      \item[2.] Adrian Aguilera Moreno: 421005200.
      \end{itemize}
  \end{minipage}}
\end{center}

\end{document}
