\documentclass{article}

% Symbols
\usepackage{amsfonts, amsthm}
\usepackage{upgreek}
\usepackage{physics}
\usepackage{cancel}
\usepackage{amssymb, latexsym, amsmath}

%Algorithms
\usepackage[ruled,lined,linesnumbered,commentsnumbered]{algorithm2e}

%% Identación
\setlength{\parindent}{0cm}

% Código
\newcommand{\code}[1]{\textcolor{white!25!black}{\texttt{#1}}}
\usepackage{listings}

%AMS
\usepackage{amsthm}
\newtheorem{algo-thm}{Algoritmo}

% Graphics
\usepackage{graphicx}
\usepackage{pgf}

% Margins
\addtolength{\voffset}{-1.5cm}
\addtolength{\hoffset}{-1.5cm}
\addtolength{\textwidth}{3cm}
\addtolength{\textheight}{3cm}

%Header-Footer
\usepackage{fancyhdr}
\renewcommand{\headrulewidth}{1pt}

\newcommand{\set}[1]{
  \left\{ #1 \right\}
}

\footskip = 50pt
\renewcommand{\headrulewidth}{1pt}

\pagestyle{fancyplain}

\begin{document}
\title{UNIVERSIDAD NACIONAL AUT\'ONOMA DE M\'EXICO\\ Facultad de Ciencias}
\author{Integrantes:\\
  Marco Silva Huerta\\
  Adri\'an Aguilera Moreno\\}
\date{}
\maketitle
\begin{center}
  \includegraphics[scale=0.20]{../Imagen/Portada.jpg}\\[0.4cm]
  \Large
  \bf{Lógica Computacional}
  \normalsize
\end{center}
\newpage
\fancyhead[r]{ Lógica Computacional 2022-2}
%%%%%%%%%%%%%%%%%%%%%%%%%%%%%%%%%%%%%%%%%%%%%%%%%%%%%
\section*{\LARGE{Práctica 2}}
%%%%%%%%%%%%%%%%%%%%%%%%%%%%%%%%%%%%%%%%%%%%%%%%%%%%%%%%%%%%%%%%%%%%%
%%%%%%%%%%%%%%%%%%%%%%% ESPECIFICACIONES AQUÍ %%%%%%%%%%%%%%%%%%%%%%%
%%%%%%%%%%%%%%%%%%%%%%%%%%%%%%%%%%%%%%%%%%%%%%%%%%%%%%%%%%%%%%%%%%%%%
\begin{enumerate}
\item \code{interp} : ésta función emplea en su caso base [de la recursión]
  a la función básica \code{elem} que implementa el \code{PRELUDE} de Haskell.
  
  Todos los demás son casos recursivos de los 5 conectivos en PROP.
\item \code{estados} : ésta función emplea a las funciones 3 y 4 en su implementación.
\item \code{vars} : se obtienen las posibles variables en la proposición y se concatenan.
\item \code{subconj} : Por medio de una lista definida por comprensión [donde la cola es
  el conjunto potenecia de la cola de la lista pasada como parámetro] se define una lista
  y se le concatena el conjunto potencia de su cola.
\item \code{modelos} : por medio de una lista por comprensión se definen las
  interpretaciones de los estados [los que sean correctos].
\item \code{tautología} : Verifica que los estados y los modelos de una proposición sean los mismos.
\item \code{satisfacen} : verifica que la interpretación sea \code{True}.
\item \code{satisf} : si los modelos no son una lista vacia [esto es, que haya al menos un modelo],
  entonces la proposición es satisfacible. En otro caso, no se encontraron modelos y no es satisfacible.
\item \code{insatisfacen} : ésta es la negación de la función 7.
\item \code{contrad} : ésta es la negación de la función 8.
\item \code{equiv} : basta ver que los estados de ambas proposiciones sean los mismos.
\end{enumerate}
Para las funciones: \code{elimEquiv, elimImpl, deMorgan}, se tienen los casos recursivos y los casos base,
pero se tiene el cuidado de realizar los cambios a mano en cuando corresponda: por ejemplo, en las leyes
de De Morgan se revisa la negación de la conjunción y de la disyunción, y es aquí donde se realiza el cambio
de conjunción a disyunción [disyunción a conjunción] y se niegan ambas proposiciones.

Las demás ya no las pudimos hacer, no nos interpretó la de \code{consecuencia} :(, dísculpa por entregar hasta
ahora, no nos vuelve a pasar.
\end{document}
