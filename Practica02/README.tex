\documentclass{article}

% Symbols
\usepackage{amsfonts, amsthm}
\usepackage{upgreek}
\usepackage{physics}
\usepackage{cancel}
\usepackage{amssymb, latexsym, amsmath}

%Algorithms
\usepackage[ruled,lined,linesnumbered,commentsnumbered]{algorithm2e}

%% Identación
\setlength{\parindent}{0cm}

% Código
\newcommand{\code}[1]{\textcolor{white!25!black}{\texttt{#1}}}
\usepackage{listings}

%AMS
\usepackage{amsthm}
\newtheorem{algo-thm}{Algoritmo}

% Graphics
\usepackage{graphicx}
\usepackage{pgf}

% Margins
\addtolength{\voffset}{-1.5cm}
\addtolength{\hoffset}{-1.5cm}
\addtolength{\textwidth}{3cm}
\addtolength{\textheight}{3cm}

%Header-Footer
\usepackage{fancyhdr}
\renewcommand{\headrulewidth}{1pt}

\newcommand{\set}[1]{
  \left\{ #1 \right\}
}

\footskip = 50pt
\renewcommand{\headrulewidth}{1pt}

\pagestyle{fancyplain}

\begin{document}
\title{UNIVERSIDAD NACIONAL AUT\'ONOMA DE M\'EXICO\\ Facultad de Ciencias}
\author{Integrantes:\\
  Marco Silva Huerta\\
  Adri\'an Aguilera Moreno\\}
\date{}
\maketitle
\begin{center}
  \includegraphics[scale=0.20]{../Imagen/Portada.jpg}\\[0.4cm]
  \Large
  \bf{Lógica Computacional}
  \normalsize
\end{center}
\newpage
\fancyhead[r]{ Lógica Computacional 2022-2}
%%%%%%%%%%%%%%%%%%%%%%%%%%%%%%%%%%%%%%%%%%%%%%%%%%%%%
\section*{\LARGE{Práctica 2}}
%%%%%%%%%%%%%%%%%%%%%%%%%%%%%%%%%%%%%%%%%%%%%%%%%%%%%%%%%%%%%%%%%%%%%
%%%%%%%%%%%%%%%%%%%%%%% ESPECIFICACIONES AQUÍ %%%%%%%%%%%%%%%%%%%%%%%
%%%%%%%%%%%%%%%%%%%%%%%%%%%%%%%%%%%%%%%%%%%%%%%%%%%%%%%%%%%%%%%%%%%%%
\begin{enumerate}
\item 
\item Para la función \code{estados} empleamos las funciones \code{vars}
  implementada en el ejercicio 3 y que obtiene la lista de todas las
  variables en una proposición, junto a \code{potenlist} que devuelve el
  conjunto potencia de la lista que se le pase. Así podemos construir todos
  los posibles estados existentes en la proposición.
\item 
\item La función 4 es realmente la función \code{potenlist} que se menciona
  en el ejercicio 2, como la necesitabamos antes, decidimos implementarla
  y \code{subconj} solo manda a llamar a \code{potenlist}.
\item 
\item Esta función solo realiza una comparación con \code{estados} y \code{modelos}.
\end{enumerate}
\end{document}
